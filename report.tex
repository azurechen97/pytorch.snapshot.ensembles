\documentclass{article}
\usepackage{graphics}
\usepackage{epsfig}
\usepackage{color}
\usepackage{times,mathptm}
\usepackage{color}
\usepackage{graphicx}
\usepackage{wasysym}
\usepackage{comment}
\usepackage{hyperref}
\usepackage[square, numbers]{natbib}
\usepackage[utf8]{inputenc}
\usepackage{geometry}

\usepackage{titling}

\usepackage[english]{babel}
\usepackage{amsthm, amsmath, amssymb}
\usepackage{float}
\usepackage{graphicx}

\geometry{margin=1in}

\setlength{\droptitle}{-8em}

\title{Final Report: Extension of Snapshot Ensemble}
\author{Aoxue Chen}
\date{December 8, 2021}

\begin{document}
\maketitle
% **** YOUR NOTES GO HERE:

% Some general latex examples and examples making use of the
% macros follow.  
%**** IN GENERAL, BE BRIEF. LONG SCRIBE NOTES, NO MATTER HOW WELL WRITTEN,
%**** ARE NEVER READ BY ANYBODY.
\section{Introduction}


\section{Goal}

what your goal was

\section{Extensions}

a detailed description of your extension of the original method

\section{Empirical Results}

empirical results

\section{Discussion}

discussion including a comparison with the original method

\section{Conclusion}

a conclusion section that summarizes your findings (which might be negative findings -- i.e. it is possible that you don't achieve your goal even after significant effort, in which case you should report what you tried anyway: you will not be penalized if, ultimately, your extension doesn't 'work')

\section*{References}
\beginrefs
\bibentry{1}{\sc Y.~Zhang}, {\sc X.~Chen}, {\sc D.~Zhou} and {\sc M.I.~Jordan},
Spectral methods meet EM: A provably optimal algorithm for crowdsourcing, 
{\it Advances in neural information processing systems\/}~{\bf 27} (2014),
pp.~1260--1268.

\noindent \medskip
\bibentry{2}{\sc A. P. Dawid}, and {\sc A. M. Skene}. Maximum likelihood estimation of observer error-rates using the EM algorithm. {\it Journal of the Royal Statistical Society, Series C\/} (1979), pp. 20-–28.
\endrefs


\end{document}